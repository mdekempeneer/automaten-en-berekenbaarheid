\begin{quest}[\emph{NFA} naar \emph{DFA}]
Beschrijf in detail de transformatie van een niet-deterministische eindige toestandsautomaat naar een equivalente deterministische eindige toestandsautomaat. Beschrijf de notie van equivalentie van automaten in deze context en argumenteer waarom de transformatie correct is. Bespreek de uitspraak ``deze transformatie is (niet) deterministisch''. Kan er zo een \emph{DFA} bestaan met minder toestanden dan de \emph{NFA}?
\end{quest}

\subsubsection*{NFA vs DFA}

Een \emph{NFA (niet-deterministische eindige automaat)} laat $\epsilon$-overgangen en meerdere bogen met hetzelfde label vanuit dezelfde knoop toe. Bij een \emph{NFA} kunnen er meerdere mogelijkheden zijn voor de volgende stap. De volgorde van de toestanden die zullen gevolgd worden ligt dus niet vast. \\
Een \emph{DFA (deterministische eindige automaat)} kan geen $\epsilon$-overgangen hebben en een symbool $\alpha \in \Sigma$ mag hoogstens op \'e\'en  uitgaande boog per toestand staan. De volgende stap van het proces, de overgang naar de volgende toestand, ligt dus vast en er is maar \'e\'en mogelijkheid.

\subsubsection*{Algoritme}

\textbf{Gegeven:} een \emph{NFA} = $(Q_n, \Sigma, \delta_n, q_{sn}, F_n)$ \\
\textbf{Gevraagd:} een \emph{DFA} = $(Q_d, \Sigma, \delta_d, q_{sd}, F_d)$ zodanig dat $L_{NFA}=L_{DFA}$.\\
\textbf{Constructie:} Vermits de \emph{NFA} $\epsilon$-bogen heeft, zullen we enkele van de toestanden van de \emph{NFA} moeten samennemen. We kunnen dus al stellen dat elke toestand in de \emph{DFA} een verzameling van toestanden van de \emph{NFA} zal zijn:
$$ Q_d =  \mathcal{P}(Q_n)$$
Verder weten we ook dat de eindtoestand van de \emph{DFA} altijd een eindtoestand van de \emph{NFA} bevat:
$$ F_d = \{S | S \in Q_d, S \cap F_n \neq \emptyset \}  $$
Nu rest er ons alleen nog de transitiefunctie $\delta_d$ om uit te werken:
\begin{itemize}
\item We weten al dat $\delta_d : (\mathcal{P} (Q_n) \times \Sigma) \rightarrow \mathcal{P}(Q_n)$.
\item We beginnen met een nieuwe afbeelding in te voeren: $eb: Q_n \rightarrow \mathcal{P}(Q_n)$. Deze functie staat voor \emph{epsilon bereikbaar}. Het resultaat van de afbeelding $eb(q)$ is dus de verzameling van toestanden in de \emph{NFA} die met nul, \'e\'en of meer $\epsilon$-bogen bereikbaar zijn vanuit \emph{q}.
\item We gaan nu de definitie van $eb$ liften naar $\mathcal{P}(Q_n)$. Voor een $\mathcal{Q} \in \mathcal{P}(Q_n)$ geldt dat: $eb(\mathcal{Q}) = \cup_{q \in \mathcal{Q}} eb(q)$.
\item We zullen $\delta_n$ liften op dezelfde manier naar $\mathcal{P}(Q_n)$. Dit wil zeggen dat voor $\mathcal{Q} \in \mathcal{P}(Q_n)$ geldt dat: $\delta_n(\mathcal{Q},a) = \cup_{q \in \mathcal{Q}}\delta_n(q,a)$ met $a \in \Sigma$.
\end{itemize}
We defini\"eren $\delta_d$ dan als volgt:
\begin{itemize}
\item Vanuit een toestand $\mathcal{Q}$ met $\mathcal{Q} \in Q_d = \{Q_{(d,1)}, Q_{(d,2)}, \dots, Q_{(d,k)}\}$  in de \emph{DFA} ga je naar de volgende toestand in de \emph{DFA} door voor elke `\emph{NFA-toestand}' in die $\mathcal{Q}$ (aangezien $\mathcal{Q} \in Q_d = \mathcal{P}(Q_n)$) eerst de overgangsfunctie van de \emph{NFA} te volgen en daarna de $\epsilon$-bogen te volgen. Van al deze resulterende toestandsverzamelingen neem je dan de unie:
$$ \delta_d(\mathcal{Q},a) = eb(\delta_n(\mathcal{Q},a)) \text{ voor elke } \mathcal{Q} \in Q_d $$
\item Tenslotte defini\"eren we nog de starttoestand van de \emph{DFA}:
$$ q_{sd} = eb(q_{sn}) $$
\end{itemize}

\subsubsection*{Equivalentie}
Twee automaten zijn equivalent als ze dezelfde taal bepalen. Het algoritme dat we zonet hebben beschreven, is in staat om alle $\epsilon$-bogen weg te werken. Deze veranderen niets aan de taal die bepaalt wordt door de automaat. De gegeven \emph{NFA} en de bekomen \emph{DFA} zijn dus equivalent ($L_{DFA} = L_{NFA}$).  De transformatie is correct aangezien de transformatie de taal bepaald door de automaat niet verandert.

We kunnen dus in het algemeen ook stellen dat twee NFA's equivalent zijn als en slechts als hun bijhorende DFA's (bekomen door de transformatie beschreven in het algoritme hierboven) equivelent zijn. Of nog: twee NFA's bepalen dezelfde taal als hun bijhorende DFA's dezelfde taal bepalen.

\subsubsection*{Deterministische transformatie}
Aangezien deze werkwijze alle mogelijke toestanden genereert, alsook alle mogelijk bogen, is er maar \'e\'en mogelijke uitkomst. Het is mogelijk om de toestanden en bogen van de \emph{DFA} op een andere volgorde te construeren, maar dit zal steeds leiden tot dezelfde \emph{DFA}.

\subsubsection*{DFA met minder toestanden dan een NFA}

Het is mogelijk om een equivalente \emph{DFA} te construeren uit een\emph{NFA} die minder toestanden bevat. Dit is bijvoorbeeld het geval wanneer er meerdere toestanden zijn die enkel met $\epsilon$-bogen verbonden zijn. De werkwijze als boven beschreven zal echter altijd meer toestanden hebben dan de \emph{NFA} aangezien $Q_d = \mathcal{P}(Q_n)$. Dit wil niet zeggen dat er naar elk van deze toestanden een boog is. Deze kan bijna altijd nog geminimaliseerd worden.
