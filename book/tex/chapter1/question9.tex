\begin{quest}[\emph{CFG} naar \emph{PDA}]
Geef het algoritme om een \emph{CFG} naar zijn \emph{PDA} om te zetten en bewijs de correctheid hiervan.
Volgens een constructie in de cursus worden meerdere elementen per keer gepusht, waarom mag dit?
\end{quest}

\subsubsection*{Algoritme}

We stellen de \emph{PDA} op door middel van drie toestanden. De starttoestand $q_s$, de eindtoestand $q_f$ en een hulptoestand $x$. We trekken slechts \'e\'en boog van $q_s$ naar $x$: die kijkt niet naar de string of stapel, en zetten een marker $\$$ op de stapel, samen met het beginsymbool (meestal $E$). Er is ook slechts \'e\'en boog van $x$ naar $q_f$: die consumeert niks van de string en haalt de marker $\$$ van de stapel. De andere bogen gaan van $x$ naar $x$, de labels corresponderen met:
\begin{enumerate}
	\item de symbolen uit het invoeralfabet: voor elke $\alpha \in \Sigma$, is er een boog met label $\alpha, \alpha \rightarrow \epsilon$; die bogen beteken dus: als de top van de stapel gelijk is aan het eerste symbool van de string, consumeer dan beide.
	\item de regels van de grammatica: voor elke regel $X \rightarrow \gamma$ is er een label $\epsilon,X \rightarrow \gamma$; die bogen betekenen dus: als de top van de stapel een niet-eindsymbool $X$ is, vervang het door de rechterkant $\gamma$ van een regel in de grammatica waarvan $X$ de linkerkant is; $\gamma$ is een rij eind- en niet-eindsymbolen.
\end{enumerate}

\subsubsection*{Correctheid}

Het algoritme is steeds correct, aangezien het hetzelfde werkt dan we intu\"itief zouden doen. Het vult steeds een nieuwe expressie in en verwijderd eventuele atomen.\footnote{Ik heb geen flauw idee, send me the answer.}

\subsubsection*{Meerdere elementen pushen}

Het is mogelijk om meerdere elementen in \'e\'en stap pushen, aangezien dit overeenkomt met meerdere $\epsilon, \epsilon \rightarrow \alpha$ stappen.
