
\begin{quest}[Myhill-Neroderelaties (a)]
  Geef de definitie van een Myhil-Nerode relatie over een taal \emph{L}, of zoals we noteren een \emph{MN(L)}-relatie. \\
  Bewijs vervolgens dat een \emph{MN(L)}-relatie bestaat als en slechts als \emph{L} regulier is. Bestaat er voor een taal \emph{L} soms meer dan één \emph{MN(L)}-relatie? Hoe zit het met \emph{MN(L)}-relaties bij \emph{PDA’s}?
\end{quest}

\subsubsection*{Definitie}

\begin{theorem}[Myhill-Nerode Relatie]
  Wanneer een equivalentierelatie $\sim_{DFA}$ voldoet aan de volgende voorwaarden:
  \begin{enumerate}
    \item $\forall x, y \in \Sigma^*, a \in \Sigma : x \sim_{DFA} y \rightarrow xa \sim_{DFA} ya$ (m.a.w. rechts congruent)
    \item $\sim_{DFA}$ verfijnt $\sim_L$ (m.a.w. $x \sim_{DFA} y \rightarrow x \sim_L y$)
    \item $\sim_{DFA}$ heeft een eindige index (m.a.w. het aantal equivalentieklassen van $\sim_{DFA}$ is eindig)
  \end{enumerate}
  Dan spreken we van een Myhill-Nerode relatie voor $L$ (oftewel $MN(L)$) indien de \emph{DFA} de taal \emph{L} bepaalt.
\end{theorem}

Dit heeft zin aangezien de drie eigenschappen verwijzen naar \emph{L}. Hierdoor kunnen we, vertrekkend van een \emph{DFA} die \emph{L} accepteert, een $MN(L)$ relatie contrueren op $\Sigma^*$.

\subsubsection*{L is regulier}

We kunnen ook het omgekeerde doen en dus, vertrekkende uit $MN(L)$ een \emph{DFA} contrueren zodat $L_{DFA} = L$.

\begin{theorem}
  Gegeven een taal \emph{L} over $Sigma$ en een $MN(L)$-relatie $\sim$ op $\Sigma^*$, dan definieert $(Q,\Sigma,\delta,q_s,F)$ een \emph{DFA} die \emph{L} bepaalt, waarbij
  \begin{enumerate}
    \item $Q = \{x_{\sim}|x \in \Sigma^*\}$\footnote{Hier is dus $x_{\sim}$ de toestand waar alle strings $x$ (of equivalent aan $x$) in terecht komen.}
    \item $q_s = \epsilon_{\sim}$
    \item $F = \{x_{\sim}|x \in L\}$
    \item $\delta(x_{\sim},a) = (xa)_{\sim}$\footnote{We hebben $x_{\sim}$ in punt 1 gedefinieerd als $\delta^*(q_s,x)$ dus deze is gelijk aan $\delta^*(q_s,xa) = (xa)_{\sim}$.}
  \end{enumerate}
\end{theorem}

\begin{proof}
  Dat $\delta$ goed gedefinieerd is, kan je bewijzen door gebruik te maken van de rechtse confruentie van $\sim$. Verder zijn alle andere ingredi\"enten van de \emph{DFA} duidelijk, in het bijzonder ook dat $Q$ (en $F$) slechts een eindig aantal toestanden bevat. We moeten enkel nog bewijzen dat $L_{DFA} = L$. Stel dat $x$ geaccepteerd wordt door de \emph{DFA}.
  $$x \in L_{DFA} \iff \delta^{q_s, x} \in F$$
  Uit de definitie van de \emph{DFA} kunnen we zeggen dat $q_s = \epsilon_{\sim}$. Dus $\delta^*{\epsilon_{\sim} \in F}$. De toestand waaring string $x$ komt kunnen we schrijven als $x_{\sim}$ (zie punt 1). We kunnen dus het volgende besluiten.
  $$\delta^*{\epsilon_{\sim}} \in F \iff x_{\sim} \in F$$
  Aangezien dit laatste geld, kunnen we zeggen dat $x \in L$. Er bestaat dus een \emph{DFA} voor \emph{L}, dus \emph{L} is regulier.
\end{proof}

\subsubsection*{Meerde MN(L) relaties}

Dit is zeker mogelijk aangezien er ook oneindig veel (niet-isomorfe) \emph{DFA's} bestaan die taal \emph{L} aanvaarden.

\subsubsection*{Hoe zit het met een PDA?}

Het principe blijft volledig hetzelfde. Echter was het bij een \emph{DFA} genoeg om enkel ervoor te zorgen dat de toestand hetzelfde was voor twee strings $x$ en $y$ om deze als equivalent te beschouwen. Indien we specifiek naar een \emph{PDA} kijken, dan moeten we een toestand niet individueel bekijken, maar in combinatie met de stack. Twee strings $x$ en $y$ zijn dan equivalent indien zowel de toestand waarin ze eindigen, als de stack, gelijk zijn.
