\begin{quest}[Church-Rosser]
	Formuleer en bespreek de stellingen van Church-Rosser. Geef daarbij hun belang i.v.m. het baseren van een programmeertaal op lambda-calculus. Geef de relatie met de programmeertaal Haskell. Hoeveel conversieregels ken je?
\end{quest}

\subsubsection*{Church-Rosser}

\begin{theorem}[Church-Rosser I]
	Indien $E_1 \stackrel{*}{\longleftrightarrow} E_2$, dan bestaat er een expressie $E$ zodanig dat $E_1 \stackrel{*}{\longrightarrow} E$ en $E_2 \stackrel{*}{\longrightarrow} E$.
\end{theorem}

Indien het mogelijk is om via conversies $E_1$ om te vormen naar expressie $E_2$ (en vice versa, ze zijn dus equivalent), dan hebben deze expressies dezelfde normaalvorm op $\alpha$-conversie na. Hieruit volgt het volgende.

\begin{theorem}[Uniciteit van een normaalvorm]
	Geen expressie kan geconverteerd worden naar twee verschillende normaalvormen (t.t.z. twee normaalvormen die niet $\alpha$-convertibel zijn in mekaar).
\end{theorem}

\begin{proof}
	Stel $E \stackrel{*}{\longleftrightarrow} E_1$ en $E \stackrel{*}{\longleftrightarrow} E_2$ waarbij $E_1$ en $E_2$ in normaalvorm staan, dan is $E_1 \stackrel{*}{\longleftrightarrow} E_2$ en dus bestaat er volgens \emph{CRI} een expressie $F$ zodat $E_1 \stackrel{*}{\longrightarrow} F$ en $E_2 \stackrel{*}{\longrightarrow} F$. Maar vermits $E_1$ en $E_2$ geen redexen bevatten, is $E_1 = F = E_2$ op $\alpha$-conversie na.
\end{proof}
M.a.w. alle eindige reductierijen vanaf een expressie $E$ eindigen in dezelfde normaalvorm. Vooraleer we verder gaan met \emph{CRII} moeten we het volgende defini\"eren. De \emph{normaalorde} bepaalt dat de meest linkse buitenste redex eerst moet gereduceerd worden.

\begin{theorem}[Church-Rosser II]
	Indien $E \stackrel{*}{\longrightarrow} N$, met $N$ in normaalvorm, dan bestaat er een reductierij in normaalorde van $E$ naar $N$.
\end{theorem}

\subsubsection*{Invloed op programmeertaal en Haskell}

Het reduceren volgens een reductierij in normaalorde heeft zowel voor- als nadelen bij het gebruik in een programmeertaal. Wanneer de reductierij niet in normaalorde wordt toegepast, is het mogelijk dat een expressie $E$ op een snellere manier kan gereduceerd worden naar een expressie $N$ in normaalvorm.\\

Wanneer we wel belang hechten aan de reductierij in normaalorde (zoals bij Haskell), dan zijn we in staat om parti\"ele functies te maken (aka currying).

\subsubsection*{Conversieregels}

We hebben drie verschillende conversieregels gezien, namelijk $\beta$-, $\alpha$- en $\eta$-conversie. $\alpha$-conversie kan omschreven worden als naamverandering van een formele parameter. $\beta$-conversie is het toepassen van een functie en vice versa. $\eta$-reductie is de eliminatie van redundante abstracties.
