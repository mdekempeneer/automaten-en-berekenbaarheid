\begin{quest}[Recursie]
	Geef de conversieregels voor lambda-calculus, bespreek de stellingen van Church-Rosser en leg a.d.h. daarvan uit hoe je een recursieve functie in lambda-calculus kunt maken. Geef ook een voorbeeld dat we niet besproken hebben in de cursus.
\end{quest}

De regels en de stellingen van \emph{Church-Rosser} zijn in de vorige vraag al aan bod gekomen. We bekijken nu enkel recursieve functies in meer detail.\\

\begin{align*}
	Pow 3 2 &= (YH) 3 2
	\\&= H (YH) 3 2
	\\&= (\lambda f . (\lambda xy. If (= Y 0) 1 (* x (f x (- Y 1)))))(YH) 3 2
	\\&= (\lambda xy . If (= Y 0) 1 (* x ((Y H) x (- Y 1)))) 3 2
	\\&= * 3 ((Y H) 3 1)
	\\&= * 3 (H (Y H) 3 1)
	\\&= * 3 ((\lambda f . (\lambda xy . If (= Y 0) 1 (* x (f x (- Y 1)))))(Y H) 3 1)
	\\&= * 3 ((\lambda xy . If (= Y 0) 1 (* x ((YH)x(- Y 1)))) 3 1)
	\\&= * 3 (* 3 ((Y H)3 0))
	\\&= * 3 (* 3 (H(YH) 3 0))
	\\&= * 3 (* 3 ((\lambda f (\lambda xy . If (= Y 0)1(* x(f x(-Y 1)))))(Y H)3 0))
	\\&= * 3 (* 3 ((\lambda xy . If (=Y 0) 1(* x ((Y H)x(- Y 1))))3 0))
	\\&= * 3 (* 3 1)
	\\&= * 3 3
	\\&= 9
\end{align*}
