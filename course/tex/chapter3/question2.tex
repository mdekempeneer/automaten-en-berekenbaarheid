\begin{question}
	Bespreek de twee noties ($A \leq_m B$ en $A \leq_T B$) van reduceerbaarheid, hun verband en op welke manier die noties kunnen gebruikt worden om aan te tonen dat een taal (on)beslisbaar/herkenbaar is.
\end{question}

\subsubsection*{Veel-\'e\'en reductie ($\leq_m$)}

Om over te gaan naar de definitie van de reductie van talen, kunnen we best eerst de definifie van Turing-berekenbaar erbij halen (indien we dit niet doen, kunnen we zeker zijn van deze bijvraag).

\begin{theorem}[Turing-berekenbare functie]
	Een functie $f$ heet Turing berekenbaar indien er een Turingmachine bestaat die bij input $s$ uiteindelijk stopt met $f(s)$ op de band.
\end{theorem}

\begin{theorem}[Reductie van talen]
	We zeggen dat taal $L_1$ (over $\Sigma_1$) naar taal $L_2$ (over $\Sigma_2$) kan gereduceerd worden indien er een afbeelding $f$ met signatuur $\Sigma^*_1\longrightarrow \Sigma^*_2$ bestaat zodanig dat $f(L_1) \subseteq L_2$ en $f(\overline{L_1}) \subseteq \overline{L_2}$, en zodanig dat $f$ Turing-berekenbaar is. We noteren dat door $L_1 \leq_m L_2$.
\end{theorem}

Tot hiertoe is het al duidelijk wat $L_1 \leq_m L_2$ wil zeggen. Het is nu nog belangrijk om het verband met herkenbaarheid en beslisbaarheid aan te tonen.

\begin{theorem}
	Als $L_1 \leq_m L_2$ en $L_2$ is beslisbaar, dan is $L_1$ beslisbaar.
\end{theorem}

\begin{proof}
	Het is belangrijk te weten dat de functie $f$ die elementen uit $L_1$ omzet naar element uit $L_2$ Turing-berekenbaar is. Concreet wil dit zeggen dat we de mogelijkheid hebben om een turingmachine op te stellen met als in put $s_1 \in L_1$ en als output $f(s_1) \in L_2$.\\
	Neem nu dat $L_2$ beslisbaar is, met zijn beslisser $B$. We construeren  nu een machine $C$ die elementen uit $L_1$ omzet (via $f$) naar elementen uit $L_2$, waarna we de beslisser $B$ laten beslissen. Hupsa, de combinatie van $C$ en $B$ is de beslisser van $L_1$ en ook deze taal is dus ook beslisbaar.
\end{proof}

\begin{theorem}
	Als $L_1 \leq_m L_2$ en $L_2$ is herkenbaar, dan is $L_1$ herkenbaar.
\end{theorem}

\begin{proof}
	Dit bewijs werkt hetzelfde als het voorgaande, om na te gaan dat wanneer $L_2$ beslisbaar is, dat dan ook $L_1$ beslisbaar is. Hier moeten we enkel de beslisser $B$ vervangen door een herkenner $H$.
\end{proof}

\begin{theorem}
	Als $L_1 \leq_m L_2$ en $L_1$ is niet-herkenbaar, dan is $L_2$ niet-herkenbaar.
\end{theorem}

\begin{proof}
	Stel $L_1$ is niet-herkenbaar en $L_2$ wel. We hebben zonet bewezen dat als $L_2$ herkenbaar is, ook $L_1$ herkenbaar moet zijn. Contradictie.
\end{proof}

\begin{theorem}
	Als $L_1 \leq_m L_2$ en $L_1$ is niet-beslisbaar, dan is $L_2$ niet-beslisbaar.
\end{theorem}

\begin{proof}
	Stel $L_1$ is niet-beslisbaar en $L_2$ wel. We hebben zonet bewezen dat als $L_2$ beslisbaar is, ook $L_1$ beslisbaar moet zijn. Contradictie.
\end{proof}

\subsubsection*{Orakels en hi\"erarchie van beslisbaarheid ($\leq_T$)}

De tweede notatie heeft betrekking tot orakelmachines in plaats van Turingmachines. Een orakelmachine heeft een andere structuur en werking die deze in staat stelt om, onder andere, $A_{TM}$ te beslissen. Een orakelmachine heeft bezit eigenlijk een grote map van booleans, met elke boolean behorend tot een string. Elke mogelijke string is gekoppeld aan deze $0$ of $1$ waarde. Een orakel dat een taal beslist zet alle strings die tot die taal behoren op $1$, alle andere op $0$. Door een inputstring $s$ te encoderen naar de locatie van de corresponderende booleaanse waarde, kan nagegaan worden of de string tot de taal behoort of niet. Het kan dus nooit in een oneindige lus terecht komen!
\\\\
Het nadeel is echter dat dit enkel een theoretische voorstelling is, die enkel conceptueel gebruikt kan worden. Het is onmogelijk om een bitmap met booleans te implementeren voor elke bestaande string\footnote{Want dat zijn er te veel.}.

\begin{theorem}[Turingreduceerbaar]
	Een taal $A$ is Turingreduceerbaar naar taal $B$, indien $A$ beslisbaar is relatief t.o.v. $B$, t.t.z. er bestaat een orakelachine $O^B$ die $A$ beslist. De notatie is $A \leq_T B$.
\end{theorem}

Dit is inderdaad zeer gelijkend op het eerste deel van deze vraag. In plaats van een beslisser voor $B$ te hebben, die $A$ ook beslist, gebruiken we nu een orakel. Dit orakel is dan (zoals eerder vermeld) een theoretisch hulpmiddel dat we kunnen gebruiken om onze kennis toe te passen op meerdere talen. Deze kunnen we echter in realiteit niet implementeren zoals we net beschreven hebben.

\begin{theorem}
	Indien $A \leq_T B$ en $B$ is beslisbaar, dan is $A$ beslisbaar.
\end{theorem}

\begin{proof}
	De definitie zegt ons dat $A \leq_T B$ enkel geldt indien we een orakel $O^B$ hebben dat $B$ \'en $A$ beslist. Dit is dus volledig afleidbaar van de definitie.\\
	Of anders: stel dat $B$ beslisbaar is en $A$ niet. Dan hebben we een orakel $O^B$ dat (theoretisch) $B$ beslist, maar niet $A$ (want deze is niet beslisbaar). Dit is meteen een contradictie met de definitie.
\end{proof}

\begin{theorem}
	Indien $A \leq_m B$ dan is ook $A \leq_T B$.\\
	M.a.w. $\leq_m$ is fijner dan $\leq_T$.
\end{theorem}

Dit is vanzelfsprekend indien we beseffen dat het orakel (nogmaals) een theoretische uitbreiding is op de Turingmachine. We gebruiken de Turingmachines om talen te herkennen of the beslissen. Het is mogelijk zo een machine te implementeren in een taal naar keuze. Er is echter een grens op het aantal talen dat we kunnen beslissen, aangezien een aantal in een oneindige lus kunnen komen tijdens het beslissingsproces. Dit is in de praktijk een probleem. Een theoretische oplossing daarvoor is het orakel. We kunnen deze machine wel gebruiken om theoretisch verder te redeneren. Dit will zeggen dat het orakel alle talen beslist dit een Turingmachine kan besliseen, plus de talen die een turingmachine niet kan beslissen (oneindig lus). Hierdoor is $A \leq_m B$ fijner dan $A \leq_T B$.