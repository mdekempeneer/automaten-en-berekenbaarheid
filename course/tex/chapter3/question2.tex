\begin{quest}[Recursie]
	Geef de conversieregels voor lambda-calculus, bespreek de stellingen van Church-Rosser en leg a.d.h. daarvan uit hoe je een recursieve functie in lambda-calculus kunt maken. Geef ook een voorbeeld dat we niet besproken hebben in de cursus.
\end{quest}

De regels en de stellingen van \emph{Church-Rosser} zijn in de vorige vraag al aan bod gekomen. We bekijken nu enkel recursieve functies in meer detail.\\

\begin{align*}
	\text{Pow} \enskip  3 \enskip 2 \enskip &= (Y\enskip H) \enskip 3 \enskip 2
	\\&= H \enskip (Y\enskip H) \enskip 3 \enskip 2
	\\&= (\lambda f. \enskip (\lambda xy. \enskip \text{IF} \enskip (= \enskip Y \enskip 0) \enskip 1 \enskip (* \enskip x \enskip (f \enskip x \enskip (- \enskip Y \enskip 1)))))(Y\enskip H) \enskip 3 \enskip 2
	\\&= (\lambda xy. \enskip \text{IF} \enskip (= \enskip Y \enskip 0) \enskip 1 \enskip (* \enskip x \enskip ((Y \enskip H) \enskip x \enskip (- \enskip Y \enskip 1)))) \enskip 3 \enskip 2
	\\&= * \enskip 3 \enskip ((Y \enskip H) \enskip 3 \enskip 1)
	\\&= * \enskip 3 \enskip (H \enskip (Y \enskip H) \enskip 3 \enskip 1)
	\\&= * \enskip 3 \enskip ((\lambda f. \enskip (\lambda xy. \enskip \text{IF} \enskip (= \enskip Y \enskip 0) \enskip 1 \enskip (* \enskip x \enskip (f \enskip x \enskip (- \enskip Y \enskip 1)))))(Y \enskip H) \enskip 3 \enskip 1)
	\\&= * \enskip 3 \enskip ((\lambda xy. \enskip \text{IF} \enskip (= \enskip Y \enskip 0) \enskip 1 \enskip (* \enskip x \enskip ((Y\enskip H)\enskip x\enskip (- \enskip Y \enskip 1)))) \enskip 3 \enskip 1)
	\\&= * \enskip 3 \enskip (* \enskip 3 \enskip ((Y \enskip H)\enskip 3 \enskip 0))
	\\&= * \enskip 3 \enskip (* \enskip 3 \enskip (H\enskip (Y \enskip H) \enskip 3 \enskip 0))
	\\&= * \enskip 3 \enskip (* \enskip 3 \enskip ((\lambda f \enskip (\lambda xy. \enskip \text{IF} \enskip (= \enskip Y \enskip 0)\enskip 1\enskip (* \enskip x\enskip (f \enskip x(-Y \enskip 1)))))(Y \enskip H)\enskip 3 \enskip 0))
	\\&= * \enskip 3 \enskip (* \enskip 3 \enskip ((\lambda xy. \enskip \text{IF} \enskip (=Y \enskip 0) \enskip 1(* \enskip x \enskip ((Y \enskip H)\enskip x\enskip (- \enskip Y \enskip 1))))\enskip 3 \enskip 0))
	\\&= * \enskip 3 \enskip (* \enskip 3 \enskip 1)
	\\&= * \enskip 3 \enskip 3
	\\&= 9
\end{align*}
