\begin{question}
Geef een precieze formulering van het pompend lemma voor reguliere talen en een bewijs ervan. Geef voorbeelden waarbij je laat zien hoe je dat lemma gebruikt om te bewijzen dat een gegeven taal regulier is en om te bewijzen dat een gegeven taal niet regulier is.
\end{question}

We beginnen met dit alles intu\"itief te bekijken:\\
Stel je hebt een reguliere taal met oneindig veel strings. Er bestaat een \emph{DFA} voor \emph{L} met $N = \#Q$ toestanden. Neem nu een string in \emph{L} die langer is dan \emph{N} en volg het pad van starttoestand naar eindtoestand. Vermits er maar \emph{N} toestanden zijn en je string meer dan \emph{N} karakters lang is en je bij elke overgang juist \'e\'en symbool achterlaat, moet je op je weg naar de eindtoestand minstens \'e\'en toestand $S$ twee of meer keer tegenkomen. Je hebt dus ergens een kring gemaakt. In deze string werd een substring van de oorspronkelijke string gebruikt.\\
We defini\"eren nu het pompend lemma voor reguliere talen.

\begin{theorem}[Pompend lemma voor reguliere talen]
Voor een reguliere taal L bestaat een pomplengte $d$, zodanig dat als $s \in L$ en $|s| \geq d$, dan bestaat er een verdeling van $s$ in stukken $x$, $y$ en $z$ zodanig dat $s = xyz$ en:
\begin{enumerate}
\item $\forall i \geq 0 : xy^iz \in L$
\item $|y| > 0$
\item $|xy| \leq d$
\end{enumerate}
\end{theorem}

\begin{proof}
Neem een \emph{DFA} die \emph{L} bepaalt. Neem $d=\#Q+1$. Neem ook een willekeurige string $s = a_1a_2 \hdots a_n$ met $n \geq d$. Beschouw nu de accepterende sequentie van toestanden $q_s = q_1,q_2,\hdots,q_f$ voor $s$; die heeft een lengte strikt groter dan $d$, dus zijn er bij de eerste $d$ zeker twee toestanden gelijk (omdat er maar $d-1$ toestanden zijn).\\

Stel dat $q_i$ en $q_j$ gelijk zijn met $i<j \leq d$, dan nemen we $x=a_1a_2\hdots a_i$ en $y=a_{i+1} \hdots a_j$ en $z$ de rest van de string. Alles volgt nu direct.
\end{proof}

Met het pompend lemma kan je \emph{niet} bewijzen dat een taal regulier is. De stelling zegt immers niet dat als het pompend lemma geldt de taal sowieso regulier is. Er kunnen nog andere condities van een reguliere taal niet voldaan zijn. Je kan echter wel aantonen dat een taal niet-regulier is door een contradictie te bewijzen met de stelling hierboven.\\

Een soortgelijk bewijs heeft steeds de volgende structuur: kies een willekeurig getal als pomplengte d, bestaat er een string langer dan d waarvoor \emph{elke} opdeling pompen verhinderd dan is deze taal nier regulier. Hier volgt een voorbeeld.

\begin{proof}
Stel dat er voor $L= \{a^nb^n | n \in \mathbb{N}\}$ een pomplengte $d$ bestaat. Beschouw dan de string $s=a^db^d$. Neem een willekeurige opdeling can $s=xyz$ met $|y| > 0$ dan zijn er drie mogelijkheden:
\begin{enumerate}
\item $y$ bevat alleen a's: dan bevat $xyyz$ meer a's dan b's en zit deze string dus niet in \emph{L}.
\item $y$ is van de vorm $a^ib^j$ met $i \neq 0, j \neq 0$: dan bevat $xyyz$ niet alle a's voor de b's, en zit de string dus niet in \emph{L}.
\item $y$ bevat alleen b's: dan bevat $xz$ meer a's dan b's en zit de string dus niet in \emph{L}.
\end{enumerate}
\emph{L} kan dus niet regulier zijn.
\end{proof}
