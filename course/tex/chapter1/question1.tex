\begin{quest}[Vraag 1]
Beschrijf in detail de transformatie van een determinitstische eindige toestandsautomaat naar een equivalente deterministische eindige toestandsautomaat met een minimaal aantal toestanden. Beschrijf de notie van equivalentie van automaten in deze context en argumenteer waarom er geen kleinere equivalente deterministische eindige toestandsautomaat bestaat.
\end{quest}

\subsubsection*{Minimale DFA}

Voor een gegeven reguliere taal \emph{L} bestaan er meestal veel \emph{DFA's} die de taal bepalen. Het is belangrijk om kleine machines te maken aangezien deze effici\"enter te werk gaan. De kleinst mogelijke \emph{DFA} die de taal \emph{L} bepaalt, noemen we de minimale \emph{DFA}. Vooraleer we dieper ingaan op het algoritme om deze minimale \emph{DFA} op te stellen, moeten we de term \emph{f-equivalentie} uitleggen.

\begin{theorem}[f-equivalentie]
	Twee toestanden $p$ en $q$ zijn f-gelijk indien $\forall w \in \Sigma^*:\delta^*(p,w) \in F \rightarrow \delta^*(q,w) \in F$. Twee toestanden zijn f-verschillend indien ze niet f-gelijk zijn.
\end{theorem}

Het vinden van \emph{f-equivalente} toestanden in een \emph{DFA} is relatief simpel. Om te beginnen weten we dat elke toestand, verschillend van de aanvaardbare eindtoestanden, \emph{f-verschillend} is van de aanvaardbare eindtoestanden. Indien dit niet zo zou zijn, zouden ook de toestanden die geen eindtoestanden zijn, eindtoestanden zijn. Dit is in contradictie met wat we net aannamen.\\

Daarna kunnen we alle mogelijke koppels van toestanden overlopen en nagaan of er een string \emph{s} bestaat, die beide toestanden naar twee \emph{f-verschillende} toestanden brengen. In dit geval zijn deze toestanden ook \emph{f-verschillend}. Indien zo een string \emph{s} niet bestaat, zijn ze \emph{f-gelijk}.

\subsubsection*{Algoritme}

Het principe van het algoritme is simpel. We beginnen met het verwijderen van alle niet bereikbare toestanden. Hierna gaan we alle \textit{f-equivalente} toestanden samennemen. In wat nu volgt wordt het algoritme verder uitgelegd, er van uitgaande dat alle niet-bereikbare toestanden reeds weggelaten zijn. Maak hier zeker oefeningen op! Het is praktisch onmogelijk dit goed uit te leggen als je het niet volledig snapt of zelf niet kunt.

\paragraph{Init} Stel een graaf \emph{V} op, zonder bogen, waarvan de nodes gelijk aan de toestanden van de \emph{DFA}. Verbind nu elke node, verschillend van de accepterende eindtoestanden, met alle mogelijk eindtoestanden en label deze met $\epsilon$. Deze bogen duiden aan dat de verbonden nodes \emph{f-verschillend} zijn\footnote{Zoals eerder vermeld zijn alle niet-aanvaardbare eindtoestanden \emph{f-verschillend} met de aanvaardbare eindtoestanden.}.

\paragraph{Repeat} Ga nu alle koppels van toestanden af die niet verbonden zijn in onze net opgestelde graaf \emph{V}. Check nu, in de originele \emph{DFA}, voor elk karakter $\alpha$ in het alfabet, naar welke toestanden de toestanden uit het koppel wijzen. Zijn deze twee toestanden verbonden met elkaar in de graaf \emph{V}? Dan hebben we een conflict en voegen we een boog toe tussen de twee originele toestanden in \emph{V} met label $\alpha$. Ga door tot we geen bogen of labels meer moeten toevoegen.

\paragraph{Final} Stel de complementsgraaf op van de gevonden graaf \emph{V}. De complementsgraaf toont nu aan welke nodes (die overeenkomen met de toestanden) samen genomen kunnen worden. Wanneer bv. node $4$ en $7$ samenhoren, maken we een node $(4,7)$. Voeg nu gewoon alle bogen van de originele \emph{DFA} toe en je hebt de minimale \emph{DFA}.

\subsubsection*{Kleinere DFA dan de minimale DFA}

Aangezien we weten dat de minimale \emph{DFA}, bekomen uit het algoritme dat we zonet hebben beschreven, geen \emph{f-equivalente} toestanden heeft, kunnen we geen kleinere \emph{DFA} vinden.

\begin{theorem}
	Als $DFA_1 = (Q_1,\Sigma,\delta_1,q_s,F_1)$ een machine is zonder onbereikbare toestanden en waarin elke twee toestanden f-verschillend zijn, dan bestaat er geen machine met strikt minder toestanden die dezelfde taal bepaalt.
\end{theorem}

\begin{proof}
	Laat $DFA_1$, zoals hierboven beschreven, de volgende toestanden hebben $\{q_1,...,q_n\}$, waarbij $q_s$ de starttoestand is. Stel dat $DFA_2 = (Q_2,\Sigma,\delta_2,p_s,F_2)$ minder toestanden heeft dan $DFA_1$. Beide bepalen wel dezelfde taal \emph{L}.\\

	Vermits in $DFA_1$ elke toestand bereikbaar is, bestaan er strings $s_i$ met $i=1 \dots n$ zodanig dat $\delta_1^*(q_s,s_i)=q_i \in Q_1$. Vermits $DFA_2$ minder toestanden heeft, moet voor een $i \neq j$ er een string $s_j$ bestaan die in dezelfde toestand eindigt\footnote{Aangezien deze minder toestanden heeft en er dus meerdere strings moeten eindigen in \'e\'en toestand.}. We zeggen dus dat $\delta^*_2(p_s,s_i)=\delta^*_2(p_s,s_j)$. Vermits $q_i$ en $q_j$ \textit{f-verschillend} zijn, bestaat een string $v$ zodanig dat $\delta_1^*(q_i, v)\in F_1$ en $\delta^*_1(q_j,v) \notin F_1$ of omgekeerd.\\

	Dus ook $\delta_1^*(q_i, s_i v)\in F_1 \and \delta^*_1(q_j,s_i v) \notin F_1$ of omgekeerd. Dit betekent dat van de strings $s_i v$ en $s_j v$ de $DFA_1$ er juist \'e\'en accepteert.\\

	Maar: $\delta_2^*(p_s,s_i v) = \delta_2^*(\delta_2^*(p_s,s_i),v) = \delta_2^*(\delta_2^*(p_s,s_j),v) = \delta_2^*(p_s,s_j v)$ hetgeen betekent dat $DFA_2$ ofwel beide strings $s_iv$ en $s_jv$ accepteert, of beide verwerpt.

	Dus kunnen $DFA_1$ en $DFA_2$ niet dezelfde taal bepalen.
\end{proof}

\subsubsection*{Equivalente machines}

We kunnen deze notie van \textit{f-equivalentie} meteen in verband brengen met de equivalentie van automaten. Twee automaten zijn namelijk equivalent indien deze dezefde taal bepalen. We kunnen dus zeggen dat hun start toestanden \textit{f-equivalent} moeten zijn. Indien twee machines equivalent zijn, zijn de minimale \emph{DFA's} isomorf\footnote{We bewijzen dit in een andere vraag met behulp van Myhill-Neroderelaties.}.
