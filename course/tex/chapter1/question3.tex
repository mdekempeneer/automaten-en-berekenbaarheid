\begin{quest}[Vraag 3]
Bewijs dat een eindige toestandsautomaat altijd een taal herkent die door een reguliere expressie wordt beschreven. Doe dat door een eindige toestandsautomaat te transformeren naar een gegeneraliseerde niet-deterministische eindige toestandsautomaat met slechts twee toestanden.\\
Kan voor een \emph{PDA} ongeveer hetzelfde gedaan worden door een gegeneraliseerde \emph{PDA} op te stellen?\\
\end{quest}

Voor we hier aan beginnen defini\"eren we de \emph{GNFA} als volgt:
\begin{theorem}[GNFA]
Een \emph{GNFA} (gegeneraliseerde niet-deterministische eindige automaat) is een eindige toestandsmachine met de volgende wijzigingen en beperkingen:
\begin{enumerate}
\item Er is slechts 1 eindtoestand $\neq$ starttoestand
\item Er is juist 1 boog van de starttoestand naar elke andere toestand en er komen geen pijlen aan (buiten de startpijl)
\item Er is juist 1 boog van elke toestand naar de eindtoestand, maar er vertrekken geen pijlen vanuit de eindtoestand
\item Tussen elke andere 2 toestanden is er juist 1 boog in beide richtingen
\item Er is juist 1 boog van elke andere toestand naar zichzelf
\item De bogen hebben als label een reguliere expressie
\end{enumerate}
\end{theorem}
In wat volgt bewijzen we dat een eindige toestandsautomaat altijd een taal herkent die door een reguliere expressie wordt beschreven. We volgen volgend stappenplan:
$$ \text{NFA} \rightarrow \text{GNFA} \rightarrow \text{GNFA met 2 toestanden} \rightarrow \text{reguliere expressie} $$
In de opgave spreken ze van een ``eindige toestandsautomaat'', we gaan er hier echter vanuit dat we starten vanaf een \emph{NFA}. Dit mag vermits elke \emph{DFA} omgezet kan worden naar een \emph{NFA}.

\begin{proof}
\subsubsection*{NFA $\rightarrow$ GNFA}
We kunnen onze \emph{NFA} omzetten naar een \emph{GNFA} als volgt:
\begin{enumerate}
\item Maak een nieuwe starttoestand en een nieuwe (unieke) eindtoestand
\item Teken $\epsilon$-bogen van de nieuwe begintoestand naar de oude begintoestand (van de \emph{NFA}) en van elke oude eindtoestand (van de \emph{NFA}) naar de nieuwe eindtoestand.
\item Teken ontbrekende bogen met een $\phi$-label.
\item Als er tussen twee toestanden twee of meer parallele gerichte bogen zijn, neem die dan samen met als label de unie van de labels van de parallelle bogen.
\end{enumerate}
Deze omzetting van \emph{NFA} naar \emph{GNFA} verandert de taal niet. We doen namelijk enkel correct manipulaties van de \emph{NFA}.\\
Nu we een \emph{GNFA} geconstrueerd hebben moeten we deze reduceren naar een GNFA met 2 toestanden.

\subsubsection*{Reduceren van de GNFA}
Dit doen we door herhaaldelijk een willekeurige toestand $X$ verschillend van de start- of eindtoestand te kiezen en deze knoop de verwijderen als volgt:
\begin{enumerate}
\item Kies toestanden $A$ en $B$ zodat er bogen zijn van\footnote{De toestand $X$ moet zicht tussen de toestanden $A$ en $B$ bevinden.}:
\begin{itemize}
\item $A$ naar $B$ met label $E_4$
\item $A$ naar $X$ met label $E_1$
\item $X$ naar zichzelf met label $E_2$
\item $X$ naar $B$ met label $E_3$
\end{itemize}
\item Vervang het label op de boog van $A$ naar $B$ door $E_4 | E_1 E_2^* E_3$.
\item Doe dit voor alle koppels $A$ en $B$
\item Verwijder de knoop $X$ met alle bogen die erin toekomen of vertrekken.
\end{enumerate}
Indien er geen toestanden meer zijn, ga dan naar de volgende stap. Op dit moment hebben we een \emph{GNFA} met twee toestanden.

\subsubsection*{Bepaal de reguliere expressie}
De \emph{GNFA} heeft nu exact twee toestanden (een start- en eindtoestand) met daartussen \'e\'en boog. Deze boog heeft nu een reguliere expressie als label. Dit is de reguliere expressie die dezelfde taal bepaald als de oorspronkelijke \emph{NFA}.
\end{proof}

\subsubsection*{Extra}
In wat volgt bewijzen we nog dat de reductie met \'e\'en enkele toestand de verzameling aanvaarde strings niet verandert. Hiervoor moeten we aantonen dat indien $s$ aanvaard werd voor de reductie, deze ook na de reductie aanvaard wordt en indien $s$ niet aanvaard werd voor de reductie, deze na de reductie ook niet aanvaard wordt.\\
\subsubsection*{String $s$ wordt nog steeds aanvaard}
\begin{proof}
Als $s$ aanvaard werd door een pad zonder $X$, wordt $s$ nog steeds aanvaard. Als het pad $X$ bevat zijn er toestanden $A$ en $B$ zodat $AX^nB$ een opeenvolging van toestanden is. De reguliere expressie op bogen $AX$, $XX$ en $XB$ zijn $E_1$, $E_2$ en $E_3$ en bijgevolg kost van $A$ naar $B$ gaan langs $X$ een stuk string van de vorm $E_1  (E_2)^* E_3$. Deze reguliere expressie staat ook in de boog $AB$ van de nieuwe \emph{GNFA}.
\end{proof}
\subsubsection*{String $s$ wordt nog steeds niet aanvaard}
\begin{proof}
Als $s$ aanvaard wordt door de gereduceerde \emph{GNFA} dan bevat het acceptatiepad uiteraard alleen toestanden verschillend van $X$. Op een boog $AB$ staat de reguliere expressie $E_4 | E_1 (E_2)^* E_3$. Die gebruiken betekend een stukje string uitgeven dat voldoet aan $E_4$ of aan $E_1 (E_2)^* E_3$. Dus in de originele \emph{GNFA} komt dit overeen met ofwel de boog \emph{AB} volgen ofwel $AX$, $XX$ (een aantal keer) en $XB$.\\
Dus als de string aanvaard wordt door de gereduceerde \emph{GNFA} wordt hij ook aanvaard door de originele \emph{GNFA}.
\end{proof}

\subsubsection*{Kan dit ook met een PDA?}
Dit is mogelijk aangezien we bij gewone \emph{FSM's} labels samen nemen in een reguliere expressie. Een \emph{PDA} kan dit ook doen, aangezien we meerdere karakters in \'e\'en enkele keer op de stapel kunnen schrijven (of verwijderen).
