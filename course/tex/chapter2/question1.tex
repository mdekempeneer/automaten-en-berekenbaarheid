\begin{question}
Beschrijf in detail de transformatie van een determinitstische eindige toestandsautomaat naar een equivalente deterministische eindige toestandsautomaat met een minimaal aantal toestanden. Beschrijf de notie van equivalentie van automaten in deze context en argumenteer waarom er geen kleinere equivalente deterministische eindige toestandsautomaat bestaat. Kan een kleinere equivalente niet-deterministische eindige toestandsautomaat bestaan?
\end{question}

pagina 30 - 35
\begin{itemize}
	\item Minimale DFA uitleggen
	\item f-equivalentie uitleggen
	\item algoritme
	\item Extra: bewijs van eindigheid
	\item Isomorfe dfa's en equivalentie
	\item geen kleinere indien alles f-verschillend is
\end{itemize} \todo{TODO}