\begin{question}
Geef een precieze formulering van het pompend lemma voor context-vrije talen en een bewijs ervan. Geef voorbeelden waarbij je laat zien hoe je dat lemma gebruikt om te bewijzen dat een gegeven taal context-vrij is en om te bewijzen dat een gegeven taal niet context-vrij is.
\end{question}

Het pompend lemma voor contextvrije talen gaat als volgt:
\begin{theorem}[Pompend lemma voor contextvrije talen]
Voor een contextvrije taal L bestaat een getal P (de pomplengte) zodanig dat elke string s van L met lengte minstens p kan opgedeeld worden in 5 stukken u,v,x,y en z uit $\Sigma^*$ zodanig dat s = uvxyz en 
\begin{enumerate}
\item $\forall i \geq 0 : uv^ixy^iz \in L$
\item $|vy| > 0$
\item $|vxy| \leq p$
\end{enumerate}
\end{theorem}

Dit is te bewijzen als volgt:\todo{TODO}
\begin{proof}

\end{proof}

