\begin{quest}[Vraag 10]
  Geef de definitie van een PDA. Leg de constructie uit van een PDA voor een gegeven contextvrije taal. Beargumenteer de correctheid.
  Is die PDA deterministisch?\\
  Wat is deterministisch precies?\\
  Wanneer wordt een string door een PDA aanvaardt? En door een CFG?\\
  Leg uit ambiguiteit van de CFG.\\
  Leidt ambiguiteit altijd tot een niet-deterministische PDA? En vice-versa?
\end{quest}

\subsubsection*{Definitie}

\begin{theorem}
  Een push-down automaat is een 6-tal $(Q,\Sigma, \Gamma, \delta, q_s, F)$ waarbij
  \begin{itemize}
    \item Q is een eindige verzameling toestanden
    \item $\sigma$ is een eindig inputalfabet
    \item $\Gamma$ is een eindig stapelalfabet
    \item $\delta$ is een overgangsfunctie met signatuur $Q \times \Sigma_\epsilon \times \Gamma_\epsilon \rightarrow \mathcal{P}(Q \times \Gamma_\epsilon)$
    \item $q_s$ is de starttoestand
    \item $F \subseteq Q$ is een verzameling eindtoestanden
  \end{itemize}
\end{theorem}

\subsubsection*{Constructie}

Zie antwoord vorige vraag.

\subsubsection*{Determinisme}

Standaard is een PDA niet deterministisch aangezien het mogelijk is dat er meerdere mogelijkheden zijn. Zo kan er ook een $\epsilon, \epsilon \rightarrow \epsilon$ boog bestaan. Een string wordt aanvaard indien deze eindigt in een aanvaardbare eindtoestand.

\subsubsection*{Ambiguiteit}

Ambiguiteit wil zeggen dan de CFG ambigu is, m.a.w. het is mogelijk een keuze te maken om na te gaan of een string wordt geaccepteerd of niet. Aangezien deze keuze ook moet reflecteren in de corresponderende PDA, is de PDA dus ook niet deterministisch.
