\documentclass[10pt,a4paper]{article}

\usepackage[utf8]{inputenc}
\usepackage[dutch]{babel}
\usepackage{amsmath}
\usepackage{amsfonts}
\usepackage{amssymb}
\usepackage{amsthm}
\usepackage{graphicx}
\usepackage{xcolor}
\usepackage{lipsum} 
\usepackage{float}
\usepackage[framemethod=default]{mdframed}

% Question command
\newtheorem{qtext}{Vraag}
\let\olddefinition\qtext
\renewcommand{\qtext}{\olddefinition\normalfont}

\setcounter{qtext}{0}
\setcounter{section}{1}

\newmdenv[skipabove=7pt,
skipbelow=7pt,
rightline=false,
leftline=false,
topline=true,
bottomline=false,
linecolor=gray,
backgroundcolor=black!8,
innerleftmargin=5pt,
innerrightmargin=5pt,
innertopmargin=0pt,
leftmargin=0cm,
rightmargin=0cm,
linewidth=2pt,
innerbottommargin=5pt]{qbox}

\newenvironment{question}{\newpage\begin{qbox}\begin{qtext}}{\end{qtext}\end{qbox}}


\newtheorem{ttext}{Definitie}

% Theorem box
\newmdenv[skipabove=7pt,
skipbelow=7pt,
backgroundcolor=black!2,
linecolor=black,
rightline=false,
leftline=true,
topline=false,
bottomline=false,
innerleftmargin=5pt,
innerrightmargin=5pt,
innertopmargin=0pt,
leftmargin=0cm,
rightmargin=0cm,
linewidth=1pt,
innerbottommargin=5pt]{tbox}

\newenvironment{theorem}{\begin{tbox}\begin{ttext}}{\end{ttext}\end{tbox}}


\newenvironment{pushcenter}{\vspace{1mm}\begin{center}}{\vspace{1mm}\end{center}}

\setlength\parindent{0pt}

\title{AUTOMATA \\ \& \\ COMPUTABILITY}
\author{\emph{Jensen Bernard}}

\begin{document}

\maketitle

\section*{Voorwoord}

Dit document bevat mogelijke examenvragen voor het vak \emph{Automaten en Berekenbaarheid}\footnote{Course G0P84a and G0P85a}, gedoceerd aan de Katholieke Universiteit Leuven. In geen enkel geval wil dit document een vervanging zijn voor de cursus. De cursustekst, \emph{Automaten en berekenbaarheid}, geschreven door \emph{Bart Demoen}, is zeer goed en het is zeker aangeraden deze grondig door te nemen voor u begint aan de volgende vraagstukken.
Ik heb dit document opgesteld tijdens het studeren van het vak, om op deze manier een overzicht te hebben van mogelijke examenvragen die we kunnen verwachten in Januari 2016. Vele vragen uit verschillende jaren komen zeer sterk overeen, daarom kan het zeker geen kwaad om deze extra aandacht te geven.
\\

Het is ook mogelijk dat er verwezen wordt naar delen uit de cursus. De meeste van deze verwijzingen zijn geschreven in December 2015. De meest recente versie was op dit moment de uitgave van 2013. Indien een nieuwere versie beschikbaar is, is het mogelijk dat de pagina's niet meer overeenstemmen. Aarzal niet om deze, zowel als mogelijke inhoudelijke fouten, aan te geven of aan te passen op Github.
\\

Ik heb vele studenten gehoord die vaak het nut niet inzien van dit vak, of dit totaal niet interessant vinden. Het is belangrijk eerst het voorwoord in de cursus eens te lezen, om een goed beeld te hebben van waar we nu eigenlijk mee bezig zijn. Indien u nog steeds van mening bent dat dit een enorm saai vak is, dan raad ik aan om ca. 2u te pauzeren om \emph{The Imitation Game}\footnote{Een film over Alan Turing, 2014} te kijken. Kom daarna terug en alles zal veel interessanter lijken dan voordien.

\newpage

\section{Talen en Automaten}

De introductie van dit hoofdstuk is op dit moment nog niet geschreven. Wanneer alle vraagstukken uit \emph{Talen en Automaten} klaar zijn, zal hier een overzicht komen van de leerstof die aan bod komt in dit hoofdstuk. Delen waar geen vragen over zijn gesteld maar wel gekend moeten zijn, zullen hier ook opgelijst worden. Zo kan dit hoofdstuk een volledig overzicht bieden van wat er over te kennen valt.
\newpage

\begin{question}
	Beschrijf in detail de transformatie van een determinitstische eindige toestandsautomaat naar een equivalente deterministische eindige toestandsautomaat met een minimaal aantal toestanden. Beschrijf de notie van equivalentie van automaten in deze context en argumenteer waarom er geen kleinere equivalente deterministische eindige toestandsautomaat bestaat. Kan een kleinere equivalente niet-deterministische eindige toestandsautomaat bestaan?
\end{question}

pagina 30 - 35
\begin{itemize}
	\item Minimale DFA uitleggen
	\item f-equivalentie uitleggen
	\item algoritme
	\item Extra: bewijs van eindigheid
	\item Isomorfe dfa's en equivalentie
	\item geen kleinere indien alles f-verschillend is
\end{itemize}

\newpage

\begin{question}
	Beschrijf in detail de transformatie van een niet-deterministische eindige toestandsautomaat naar een equivalente deterministische eindige toestandsautomaat. Beschrijf de notie van "equivalentie van automaten" in deze context en argumenteer waarom de transformatie correct is. Bespreek de uitspraak "deze transformatie is (niet?) deterministisch". Kan er zo een DFA bestaan met minder toestanden dan de NFA?
\end{question}

pagina 26 - 29
\begin{itemize}
	\item DFA vs NFA
	\item Algoritme
	\item isomorf vs equivalent: aangezien enkel eb weg is: equivalent
	\item ebs bogen weg doen
	\item deterministisch want eindige toestanden
	\item Ja, zie voorbeeld
\end{itemize}

\newpage

\begin{question}
	Bewijs dat een eindige toestandsautomaat altijd een taal herkent die door een reguliere expressie wordt beschreven. Doe dat door een eindige toestandsautomaat te transformeren naar een gegeneraliseerde niet-deterministische eindige toestandsautomaat met slechts twee toestanden.
\end{question}

pagina 21 - 25

\newpage

\begin{question}
	Geef een precieze formulering van het pompend lemma voor reguliere talen en een bewijs ervan. Geef voorbeelden waarbij je laat zien hoe je dat lemma gebruikt om te bewijzen dat een gegeven taal regulier is en om te bewijzen dat een gegeven taal niet regulier is.
\end{question}

pagina 43 - 45

\newpage

\begin{question}
	
\end{question}

\section{Talen en Berekenbaarheid}

\subsection{Inleiding}

\vspace{3mm}
In hoofdstuk 3, Talen en Berekenbaarheid, zal er dieper worden ingegaan op Turingmachines en de werking ervan. Na het instuderen van dit hoofdstuk is het best om onderstaande vragen op te lossen. Het zijn niet zo veel vragen, maar ze bevatten steeds een redelijk groot deel van de leerstof (vaak ook verschillende stukken gecombineerd). De vragen die volgen zijn alle vragen die al eens gesteld zijn. Deze komen echter elk jaar terug. Concreet wil dit zeggen dat als al deze vragen gekend zijn een goed resultaat verwacht kan worden.
\\\\
Dit wil echter niet zeggen dat het onmogelijk is om een andere vraag te krijgen. In de volgende sectie staat een kleine opsomming van delen uit het hoofdstuk die (tot nu toe) niet aan bod gekomen zijn tijdens de examens\footnote{Het gaat hier echter enkel over het mondeling examen, ze kunnen wel voorkomen op de tussentijds testen.}. Er bestaat echter een kleine kans dat hij zijn vragen veranderd, wat wil zeggen dat ook deze delen gekend moeten zijn. In het algemeen zijn de vragen uit dit document echter goed genoeg.
\\\\
Achteraan dit document zal ook verwezen worden naar een implementatie van een Turingmachine in Haskell. Dit kan een dieper inzicht geven op hoe zo een machine werkt.
\\\\
Aarzel niet om fouten of verbeteringen te melden op Github\footnote{User: Jense5}.

\subsection{Overige secties}

De volgende secties uit hoofdstuk 3 van Automaten en Berekenbaarheid zijn tot nu toe nog nooit aan bod gekomen op het examen, maar moeten wel gekend zijn. De pagina's komen overeen met de publicatie van 19 november 2013\footnote{De meeste recente versie in 2015.}.
\begin{enumerate}
	\item Basiswerking van een Turingmachine (p.82-86)
	\item Er bestaat een niet herkenbare taal (p.87)
	\item Universele Turingmachines (p.95)
	\item Verband met reguliere talen (p.100)
	\item $Regular_{TM}$ en $EQ_{TM}$ (p.105)
	\item Aftelbaar (p.110)
	\item The Post Correspondence Problem (p.113-116)
	\item Recursieve functies (p.122-125)
	\item De bezige bever (p.126-127)
\end{enumerate}

\begin{question}
	Formuleer en bespreek de stellingen van Church-Rosser. Geef daarbij hun belang i.v.m. het baserenvan een programmeertaal op lambda-calculus. Geef de relatie met de programmeertaal Haskell. Hoeveel conversieregels ken je?
\end{question}

\lipsum[1-2]
\begin{question}
	Bespreek de twee noties ($A \leq_m B$ en $A \leq_T B$) van reduceerbaarheid, hun verband en op welke manier die noties kunnen gebruikt worden om aan te tonen dat een taal (on)beslisbaar/herkenbaar is.
\end{question}

\subsubsection*{Veel-\'e\'en reductie ($\leq_m$)}

Om over te gaan naar de definitie van de reductie van talen, kunnen we best eerst de definifie van Turing-berekenbaar erbij halen (indien we dit niet doen, kunnen we zeker zijn van deze bijvraag).

\begin{theorem}[Turing-berekenbare functie]
	Een functie $f$ heet Turing berekenbaar indien er een Turingmachine bestaat die bij input $s$ uiteindelijk stopt met $f(s)$ op de band.
\end{theorem}

\begin{theorem}[Reductie van talen]
	We zeggen dat taal $L_1$ (over $\Sigma_1$) naar taal $L_2$ (over $\Sigma_2$) kan gereduceerd worden indien er een afbeelding $f$ met signatuur $\Sigma^*_1\longrightarrow \Sigma^*_2$ bestaat zodanig dat $f(L_1) \subseteq L_2$ en $f(\overline{L_1}) \subseteq \overline{L_2}$, en zodanig dat $f$ Turing-berekenbaar is. We noteren dat door $L_1 \leq_m L_2$.
\end{theorem}

Tot hiertoe is het al duidelijk wat $L_1 \leq_m L_2$ wil zeggen. Het is nu nog belangrijk om het verband met herkenbaarheid en beslisbaarheid aan te tonen.

\begin{theorem}
	Als $L_1 \leq_m L_2$ en $L_2$ is beslisbaar, dan is $L_1$ beslisbaar.
\end{theorem}

\begin{proof}
	Het is belangrijk te weten dat de functie $f$ die elementen uit $L_1$ omzet naar element uit $L_2$ Turing-berekenbaar is. Concreet wil dit zeggen dat we de mogelijkheid hebben om een turingmachine op te stellen met als in put $s_1 \in L_1$ en als output $f(s_1) \in L_2$.\\
	Neem nu dat $L_2$ beslisbaar is, met zijn beslisser $B$. We construeren  nu een machine $C$ die elementen uit $L_1$ omzet (via $f$) naar elementen uit $L_2$, waarna we de beslisser $B$ laten beslissen. Hupsa, de combinatie van $C$ en $B$ is de beslisser van $L_1$ en ook deze taal is dus ook beslisbaar.
\end{proof}

\begin{theorem}
	Als $L_1 \leq_m L_2$ en $L_2$ is herkenbaar, dan is $L_1$ herkenbaar.
\end{theorem}

\begin{proof}
	Dit bewijs werkt hetzelfde als het voorgaande, om na te gaan dat wanneer $L_2$ beslisbaar is, dat dan ook $L_1$ beslisbaar is. Hier moeten we enkel de beslisser $B$ vervangen door een herkenner $H$.
\end{proof}

\begin{theorem}
	Als $L_1 \leq_m L_2$ en $L_1$ is niet-herkenbaar, dan is $L_2$ niet-herkenbaar.
\end{theorem}

\begin{proof}
	Stel $L_1$ is niet-herkenbaar en $L_2$ wel. We hebben zonet bewezen dat als $L_2$ herkenbaar is, ook $L_1$ herkenbaar moet zijn. Contradictie.
\end{proof}

\begin{theorem}
	Als $L_1 \leq_m L_2$ en $L_1$ is niet-beslisbaar, dan is $L_2$ niet-beslisbaar.
\end{theorem}

\begin{proof}
	Stel $L_1$ is niet-beslisbaar en $L_2$ wel. We hebben zonet bewezen dat als $L_2$ beslisbaar is, ook $L_1$ beslisbaar moet zijn. Contradictie.
\end{proof}

\subsubsection*{Orakels en hi\"erarchie van beslisbaarheid ($\leq_T$)}

De tweede notatie heeft betrekking tot orakelmachines in plaats van Turingmachines. Een orakelmachine heeft een andere structuur en werking die deze in staat stelt om, onder andere, $A_{TM}$ te beslissen. Een orakelmachine heeft bezit eigenlijk een grote map van booleans, met elke boolean behorend tot een string. Elke mogelijke string is gekoppeld aan deze $0$ of $1$ waarde. Een orakel dat een taal beslist zet alle strings die tot die taal behoren op $1$, alle andere op $0$. Door een inputstring $s$ te encoderen naar de locatie van de corresponderende booleaanse waarde, kan nagegaan worden of de string tot de taal behoort of niet. Het kan dus nooit in een oneindige lus terecht komen!
\\\\
Het nadeel is echter dat dit enkel een theoretische voorstelling is, die enkel conceptueel gebruikt kan worden. Het is onmogelijk om een bitmap met booleans te implementeren voor elke bestaande string\footnote{Want dat zijn er te veel.}.

\begin{theorem}[Turingreduceerbaar]
	Een taal $A$ is Turingreduceerbaar naar taal $B$, indien $A$ beslisbaar is relatief t.o.v. $B$, t.t.z. er bestaat een orakelachine $O^B$ die $A$ beslist. De notatie is $A \leq_T B$.
\end{theorem}

Dit is inderdaad zeer gelijkend op het eerste deel van deze vraag. In plaats van een beslisser voor $B$ te hebben, die $A$ ook beslist, gebruiken we nu een orakel. Dit orakel is dan (zoals eerder vermeld) een theoretisch hulpmiddel dat we kunnen gebruiken om onze kennis toe te passen op meerdere talen. Deze kunnen we echter in realiteit niet implementeren zoals we net beschreven hebben.

\begin{theorem}
	Indien $A \leq_T B$ en $B$ is beslisbaar, dan is $A$ beslisbaar.
\end{theorem}

\begin{proof}
	De definitie zegt ons dat $A \leq_T B$ enkel geldt indien we een orakel $O^B$ hebben dat $B$ \'en $A$ beslist. Dit is dus volledig afleidbaar van de definitie.\\
	Of anders: stel dat $B$ beslisbaar is en $A$ niet. Dan hebben we een orakel $O^B$ dat (theoretisch) $B$ beslist, maar niet $A$ (want deze is niet beslisbaar). Dit is meteen een contradictie met de definitie.
\end{proof}

\begin{theorem}
	Indien $A \leq_m B$ dan is ook $A \leq_T B$.\\
	M.a.w. $\leq_m$ is fijner dan $\leq_T$.
\end{theorem}

Dit is vanzelfsprekend indien we beseffen dat het orakel (nogmaals) een theoretische uitbreiding is op de Turingmachine. We gebruiken de Turingmachines om talen te herkennen of the beslissen. Het is mogelijk zo een machine te implementeren in een taal naar keuze. Er is echter een grens op het aantal talen dat we kunnen beslissen, aangezien een aantal in een oneindige lus kunnen komen tijdens het beslissingsproces. Dit is in de praktijk een probleem. Een theoretische oplossing daarvoor is het orakel. We kunnen deze machine wel gebruiken om theoretisch verder te redeneren. Dit will zeggen dat het orakel alle talen beslist dit een Turingmachine kan besliseen, plus de talen die een turingmachine niet kan beslissen (oneindig lus). Hierdoor is $A \leq_m B$ fijner dan $A \leq_T B$.
\begin{question}
Bewijs dat een eindige toestandsautomaat altijd een taal herkent die door een reguliere expressie wordt beschreven. Doe dat door een eindige toestandsautomaat te transformeren naar een gegeneraliseerde niet-deterministische eindige toestandsautomaat met slechts twee toestanden.\\
Kan voor een PDA ongeveer hetzelfde gedaan worden door een gegeneraliseerde PDA op te stellen?\\
\end{question}

Voor we hier aan beginnen, defini\"eren we de \emph{GNFA} als volgt:
\begin{theorem}[GNFA]
Een GNFA (gegeneraliseerde niet-deterministische eindige automaat) is een eindige toestandsmachine met de volgende wijzigingen en beperkingen:
\begin{enumerate}
\item Er is slechts 1 eindtoestand $\neq$ starttoestand
\item Er is juist 1 boog van de starttoestand naar elke andere toestand en er komen geen pijlen aan (buiten de startpijl)
\item Er is juist 1 boog van elke toestand naar de eindtoestand, maar er vertrekken geen pijlen vanuit de eindtoestand
\item Tussen elke andere 2 toestanden is er juist 1 boog in beide richtingen
\item Er is juist 1 boog van elke andere toestand naar zichzelf
\item De bogen hebben als label een reguliere expressie
\end{enumerate}
\end{theorem}
In wat volgt bewijzen we dat een eindige toestandsautomaat altijd een taal herkent die door een reguliere expressie wordt beschreven. We volgen volgend stappenplan:
$$ \text{NFA} \rightarrow \text{GNFA} \rightarrow \text{GNFA met 2 toestanden} \rightarrow \text{reguliere expressie} $$
In de opgave spreken ze van een ``eindige toestandsautomaat'', we gaan er hier echter vanuit dat we starten vanaf een NFA. Dit mag vermits elke DFA omgezet kan worden naar een NFA.

\begin{proof}
\subsubsection*{(a) NFA $\rightarrow$ GNFA}
We kunnen onze NFA omzetten naar een GNFA als volgt:
\begin{enumerate}
\item Maak een nieuwe starttoestand en een nieuwe (unieke) eindtoestand
\item Teken $\epsilon$-bogen van de nieuwe begintoestand naar de oude begintoestand en van elke oude eindtoestand naar de nieuwe eindtoestand.
\item Teken ontbrekende bogen met een $\phi$.
\item Als er tussen twee toestanden twee of meer parallele gerichte bogen zijn, neem die dan samen met als label de unie van de labels van de parallelle bogen.
\end{enumerate}
Deze omzetting van NFA naar GNFA verandert de taal niet. We doen namelijk enkel correct manipulaties van de NFA.\\
Nu we een GNFA geconstrueerd hebben moeten we deze reduceren naar een GNFA met 2 toestanden.

\subsubsection*{(b) Reduceren van de GNFA}
Dit doen we door herhaaldelijk een willekeurige toestand X verschillend van de start- of eindtoestand te kiezen en deze knoop de verwijderen als volgt:
\begin{enumerate}
\item Kies toestanden A en B zodat er bogen zijn van: 
\begin{itemize}
\item A naar B met label $E_4$
\item A naar X met label $E_1$
\item X naar zichzelf met label $E_2$
\item X naar B met label $E_3$
\end{itemize}
\item Vervang het label op de boog van A naar B door $E_4 | E_1 E_2^* E_3$.
\item Doe dit voor alle koppels A en B
\item Verwijder de knoop X met alle bogen die erin toekomen of vertrekken.
\end{enumerate}
Indien er geen toestanden meer zijn, ga naar stap (c).

\subsubsection*{(c) Bepaal RE}
De GNFA heeft nu exact 2 toestanden (een start- en eindtoestand) met daartussen 1 boog. Deze boog heeft nu een reguliere expressie als label. Dit is de reguliere expressie die dezelfde taal bepaald als de oorspronkelijke NFA.
\end{proof}

\subsubsection*{(d) Extra}
In wat volgt bewijzen we nog dat de reductie met 1 toestand de verzameling aanvaarde strings niet verandert. Hiervoor moeten we aantonen dat indien s aanvaard werd voor de reductie, deze ook na de reductie aanvaard wordt (1) en indien s niet aanvaard werd voor de reductie, deze na de reductie ook niet aanvaard wordt (2).\\
\subsubsection*{(1)}
\begin{proof}
Als s aanvaard werd door een pad zonder X, wordt s nog steeds aanvaard: paden zonder X blijven namelijk bestaan. Als het pad X bevat zijn er toestanden A en B zodat $AX^n B$ een opeenvolging van toestanden is. De reguliere expressie op bogen AX,XX en XB zijn $E_1$, $E_2$ en $E_3$ en bijgevolg kost van A naar B gaan langs X een stuk string van de vorm $E_1  (E_2)^* E_3$. Deze reguliere expressie staat ook in de boog AB van de nieuwe GNFA.
\end{proof}
\subsubsection*{(2)}
\begin{proof}
Als s aanvaard wordt door de gereduceerde GNFA dan bevat het acceptatiepad uiteraard alleen toestanden verschillend van X. Op een boog AB staat de reguliere expressie $E_4 | E_1 (E_2)^* E_3$. Die gebruiken betekend een stukje string uitgeven dat voldoet aan $E_4$ of aan $E_1 (E_2)^* E_3$. Dus in de originele GNFA komt dit overeen met ofwel de boog AB volgen ofwel AX,XX (een aantal keer) en XB.\\
Dus als de string aanvaard wordt door de gereduceerde GNFA wordt hij ook aanvaard door de originele GNFA.
\end{proof}

\subsubsection*{(e) Kan dit ook met een PDA?}
Dit is mogelijk aangezien we bij gewone FSM's labels samen nemen in een reguliere expressie. Een PDA kan dit ook doen, aangezien we meerdere karakters in \'e\'en enkele keer op de stapel kunnen schrijven (of verwijderen).

\begin{question}
Geef een precieze formulering van het pompend lemma voor reguliere talen en een bewijs ervan. Geef voorbeelden waarbij je laat zien hoe je dat lemma gebruikt om te bewijzen dat een gegeven taal regulier is en om te bewijzen dat een gegeven taal niet regulier is.
\end{question}

We beginnen met dit alles intu\"itief te bekijken:\\
Stel je hebt een reguliere taal met oneindig veel strings. Er bestaat een DFA voor L met $N = \#Q$ toestanden. Neem nu een string in L die langer is dan N en begin de tocht van starttoestand naar eindtoestand. Vermits er maar N toestanden zijn en je string meer dan N lang is en je bij elke overgang juist 1 symbool achterlaat, moet je op je weg naar de eindtoestand minstens 1 toestand S 2 of meer keer tegenkomen. Je hebt dus ergens een kring gemaakt. In deze string werd een substring van de oorspronkelijke string gebruikt.\\
We defini\"eren nu het pompend lemma voor reguliere talen.

\begin{theorem}[Pompend lemma voor reguliere talen]
Voor een reguliere taal L bestaat een pomplengte $d$, zodanig dat als $s \in L$ en $|s| \geq d$, dan bestaat er een verdeling van $s$ in stukken x,y en z zodanig dat s = xyz en:
\begin{enumerate}
\item $\forall i \geq 0 : xy^iz \in L$
\item $|y| > 0$
\item $|xy| \leq d$
\end{enumerate}
\end{theorem}

In wat volgt bewijzen we deze stelling:

\begin{proof}
Neem een DFA die L bepaalt. Neem d = $\#Q+1$.\\
Neem een willekeurige string $s = a_1a_2 \hdots a_n$ met $n \geq d$. \\
Beschouw de accepterende sequentie van toestanden ($q_s = q_1,q_2,\hdots,q_f)$ voor $s$; die heeft lengte strikt groter dan $d$, dus zijn er bij de eerste $d$ zeker twee toestanden gelijk (omdat er maar $d-1$ toestanden zijn).\\
Stel dat $q_i$ en $q_j$ gelijk zijn met $i<j \leq d$ dan nemen we $x=a_1a_2\hdots a_i$ en $y=a_{i+1} \hdots a_j$ en $z$ de rest van de string. Alles volgt nu direct.
\end{proof}

Met het pompend lemma kan je \emph{niet} bewijzen dat een taal regulier is. De stelling zegt immers niet dat als het pompend lemma geldt de taal sowieso regulier is. Er kunnen nog andere condities van een reguliere taal niet voldaan zijn.
Je kan echter wel aantonen dat een taal niet-regulier is door een contradictie te bewijzen met de stelling hierboven.\\
Een soortgelijk bewijs heeft steeds de volgende structuur: kies een willekeurig getal als pomplengte d, bestaat er een string langer dan d waarvoor \emph{elke} opdeling pompen verhinderd dan is deze taal nier regulier.

Algemeen voorbeeld:
\begin{proof}
Stel dat er voor $L= \{a^nb^n | n \in \mathbb{N}\}$ een pomplengte $d$ bestaat. Beschouw dan de string $s=a^db^d$. Neem een willekeurige opdeling can $s=xyz$ met $|y| > 0$ dan zijn er 3 mogelijkheden:
\begin{enumerate}
\item y bevat alleen a's: dan bevat xyyz meer a's dan b's en zit deze string dus niet in $L$
\item y is van de vorm $a^ib^j$ met $i \neq 0, j \neq 0$: dan bevat xyyz niet alle a's voor de b's, en zit de string dus niet in L
\item y bevat alleen b's: dan bevat xz meer a's dan b's en zit de string dus niet in L. L kan dus niet regulier zijn.
\end{enumerate}
\end{proof}


\begin{quest}[Myhill-Neroderelaties (a)]
  Geef de definitie van een Myhil-Nerode relatie over een taal \emph{L}, of zoals we noteren een \emph{MN(L)}-relatie. \\
  Bewijs vervolgens dat een \emph{MN(L)}-relatie bestaat als en slechts als \emph{L} regulier is. Bestaat er voor een taal \emph{L} soms meer dan één \emph{MN(L)}-relatie? Hoe zit het met \emph{MN(L)}-relaties bij \emph{PDA’s}?
\end{quest}

\subsubsection*{Definitie}

\begin{theorem}[Myhill-Nerode Relatie]
  Wanneer een equivalentierelatie $\sim_{DFA}$ voldoet aan de volgende voorwaarden:
  \begin{enumerate}
    \item $\forall x, y \in \Sigma^*, a \in \Sigma : x \sim_{DFA} y \rightarrow xa \sim_{DFA} ya$ (m.a.w. rechts congruent)
    \item $\sim_{DFA}$ verfijnt $\sim_L$ (m.a.w. $x \sim_{DFA} y \rightarrow x \sim_L y$)
    \item $\sim_{DFA}$ heeft een eindige index (m.a.w. het aantal equivalentieklassen van $\sim_{DFA}$ is eindig)
  \end{enumerate}
  Dan spreken we van een Myhill-Nerode relatie voor $L$ (oftewel $MN(L)$) indien de \emph{DFA} de taal \emph{L} bepaalt.
\end{theorem}

Dit heeft zin aangezien de drie eigenschappen verwijzen naar \emph{L}. Hierdoor kunnen we, vertrekkend van een \emph{DFA} die \emph{L} accepteert, een $MN(L)$ relatie contrueren op $\Sigma^*$.

\subsubsection*{L is regulier}

We kunnen ook het omgekeerde doen en dus, vertrekkende uit $MN(L)$ een \emph{DFA} contrueren zodat $L_{DFA} = L$.

\begin{theorem}
  Gegeven een taal \emph{L} over $Sigma$ en een $MN(L)$-relatie $\sim$ op $\Sigma^*$, dan definieert $(Q,\Sigma,\delta,q_s,F)$ een \emph{DFA} die \emph{L} bepaalt, waarbij
  \begin{enumerate}
    \item $Q = \{x_{\sim}|x \in \Sigma^*\}$\footnote{Hier is dus $x_{\sim}$ de toestand waar alle strings $x$ (of equivalent aan $x$) in terecht komen.}
    \item $q_s = \epsilon_{\sim}$
    \item $F = \{x_{\sim}|x \in L\}$
    \item $\delta(x_{\sim},a) = (xa)_{\sim}$\footnote{We hebben $x_{\sim}$ in punt 1 gedefinieerd als $\delta^*(q_s,x)$ dus deze is gelijk aan $\delta^*(q_s,xa) = (xa)_{\sim}$.}
  \end{enumerate}
\end{theorem}

\begin{proof}
  Dat $\delta$ goed gedefinieerd is, kan je bewijzen door gebruik te maken van de rechtse confruentie van $\sim$. Verder zijn alle andere ingredi\"enten van de \emph{DFA} duidelijk, in het bijzonder ook dat $Q$ (en $F$) slechts een eindig aantal toestanden bevat. We moeten enkel nog bewijzen dat $L_{DFA} = L$. Stel dat $x$ geaccepteerd wordt door de \emph{DFA}.
  $$x \in L_{DFA} \iff \delta^{q_s, x} \in F$$
  Uit de definitie van de \emph{DFA} kunnen we zeggen dat $q_s = \epsilon_{\sim}$. Dus $\delta^*{\epsilon_{\sim} \in F}$. De toestand waaring string $x$ komt kunnen we schrijven als $x_{\sim}$ (zie punt 1). We kunnen dus het volgende besluiten.
  $$\delta^*{\epsilon_{\sim}} \in F \iff x_{\sim} \in F$$
  Aangezien dit laatste geld, kunnen we zeggen dat $x \in L$. Er bestaat dus een \emph{DFA} voor \emph{L}, dus \emph{L} is regulier.
\end{proof}

\subsubsection*{Meerde MN(L) relaties}

Dit is zeker mogelijk aangezien er ook oneindig veel (niet-isomorfe) \emph{DFA's} bestaan die taal \emph{L} aanvaarden.

\subsubsection*{Hoe zit het met een PDA?}

Het principe blijft volledig hetzelfde. Echter was het bij een \emph{DFA} genoeg om enkel ervoor te zorgen dat de toestand hetzelfde was voor twee strings $x$ en $y$ om deze als equivalent te beschouwen. Indien we specifiek naar een \emph{PDA} kijken, dan moeten we een toestand niet individueel bekijken, maar in combinatie met de stack. Twee strings $x$ en $y$ zijn dan equivalent indien zowel de toestand waarin ze eindigen, als de stack, gelijk zijn.

\begin{question}
Reconstrueer de details van het verhaal van Myhill-Nerode: elke DFA bepaalt een Myhill-Neroderelatie; elke Myhill-Neroderelatie bepaalt een DFA; twee isomorfe DFA's bepalen dezelfde Myhill-Nerode-relatie; het supremum van twee Myhill-Neroderelaties is een Myhill-Neroderelatie.
Argumenteer ook het verband tussen de minimale DFA voor een reguliere taal L en de grofste MN(L)-relatie.
Kan je ook een Myhill-Neroderelatie construeren voor een NFA? Wat krijg je dan met minimalisatie?
\end{question}\todo{TODO}
\begin{quest}[Vraag 7]
  Geef alle stappen (stellingen \& bewijzen) om aan te tonen dat alle minimale \emph{DFA's} die dezelfde taal bepalen isomorf zijn.
\end{quest}

De twee voorgaande vragen zijn belangrijk voor deze vraag. Dit antwoord baseert zich op voorgaande theori\"en. Ik zou ze niet allemaal geven, maar als hij doorvraagt moet je ze wel kennen.

\begin{proof}

  Neem twee minimale \emph{DFA's} voor eent taal \emph{L}. We zullen bewijzen dat ze isomorf zijn. Beiden hebben hetzelfde aantal toestanden $N$ - anders was \'e\'en van de twee niet minimaal. Neem nu het supremum van die twee \emph{DFA's}: je krijgt een \emph{DFA} met opnieuw $N$ toestanden, want meer kan niet, maar ook minder niet.

  In termen van de $MN(L)$-relaties door de drie betrokken \emph{DFA's} ge\"induceerd betekent dat dat die drie relaties identiek zijn. In termen van de drie \emph{DFA's} betekent het dat ze alle drie isomorf zijn.

  Pas nu weten we zeker dat de minimisatieprocedure eindigt met een uniek resultaat (op isomorfisme na).
\end{proof}

De karakterisatie van de $MN(L)$-relatie die hoort bij de minimale \emph{DFA} is redelijk eenvoudig:
$$x \sim_{min} y \iff \forall s \in \Sigma^*(xs \in L \iff ys \in L)$$
In essentie zegt dit: als twee toestanden \emph{f-gelijk} zijn, dan zijn ze identiek.


\newpage
\section{Herschrijfsystemen}

\subsection{Inleiding}

Het deel over herschrijfsystemen is relatief kort en makkelijk. Deze inleiding is nog niet klaar. :)

\begin{question}
	Formuleer en bespreek de stellingen van Church-Rosser. Geef daarbij hun belang i.v.m. het baserenvan een programmeertaal op lambda-calculus. Geef de relatie met de programmeertaal Haskell. Hoeveel conversieregels ken je?
\end{question}

\lipsum[1-2]
\begin{question}
	Bespreek de twee noties ($A \leq_m B$ en $A \leq_T B$) van reduceerbaarheid, hun verband en op welke manier die noties kunnen gebruikt worden om aan te tonen dat een taal (on)beslisbaar/herkenbaar is.
\end{question}

\subsubsection*{Veel-\'e\'en reductie ($\leq_m$)}

Om over te gaan naar de definitie van de reductie van talen, kunnen we best eerst de definifie van Turing-berekenbaar erbij halen (indien we dit niet doen, kunnen we zeker zijn van deze bijvraag).

\begin{theorem}[Turing-berekenbare functie]
	Een functie $f$ heet Turing berekenbaar indien er een Turingmachine bestaat die bij input $s$ uiteindelijk stopt met $f(s)$ op de band.
\end{theorem}

\begin{theorem}[Reductie van talen]
	We zeggen dat taal $L_1$ (over $\Sigma_1$) naar taal $L_2$ (over $\Sigma_2$) kan gereduceerd worden indien er een afbeelding $f$ met signatuur $\Sigma^*_1\longrightarrow \Sigma^*_2$ bestaat zodanig dat $f(L_1) \subseteq L_2$ en $f(\overline{L_1}) \subseteq \overline{L_2}$, en zodanig dat $f$ Turing-berekenbaar is. We noteren dat door $L_1 \leq_m L_2$.
\end{theorem}

Tot hiertoe is het al duidelijk wat $L_1 \leq_m L_2$ wil zeggen. Het is nu nog belangrijk om het verband met herkenbaarheid en beslisbaarheid aan te tonen.

\begin{theorem}
	Als $L_1 \leq_m L_2$ en $L_2$ is beslisbaar, dan is $L_1$ beslisbaar.
\end{theorem}

\begin{proof}
	Het is belangrijk te weten dat de functie $f$ die elementen uit $L_1$ omzet naar element uit $L_2$ Turing-berekenbaar is. Concreet wil dit zeggen dat we de mogelijkheid hebben om een turingmachine op te stellen met als in put $s_1 \in L_1$ en als output $f(s_1) \in L_2$.\\
	Neem nu dat $L_2$ beslisbaar is, met zijn beslisser $B$. We construeren  nu een machine $C$ die elementen uit $L_1$ omzet (via $f$) naar elementen uit $L_2$, waarna we de beslisser $B$ laten beslissen. Hupsa, de combinatie van $C$ en $B$ is de beslisser van $L_1$ en ook deze taal is dus ook beslisbaar.
\end{proof}

\begin{theorem}
	Als $L_1 \leq_m L_2$ en $L_2$ is herkenbaar, dan is $L_1$ herkenbaar.
\end{theorem}

\begin{proof}
	Dit bewijs werkt hetzelfde als het voorgaande, om na te gaan dat wanneer $L_2$ beslisbaar is, dat dan ook $L_1$ beslisbaar is. Hier moeten we enkel de beslisser $B$ vervangen door een herkenner $H$.
\end{proof}

\begin{theorem}
	Als $L_1 \leq_m L_2$ en $L_1$ is niet-herkenbaar, dan is $L_2$ niet-herkenbaar.
\end{theorem}

\begin{proof}
	Stel $L_1$ is niet-herkenbaar en $L_2$ wel. We hebben zonet bewezen dat als $L_2$ herkenbaar is, ook $L_1$ herkenbaar moet zijn. Contradictie.
\end{proof}

\begin{theorem}
	Als $L_1 \leq_m L_2$ en $L_1$ is niet-beslisbaar, dan is $L_2$ niet-beslisbaar.
\end{theorem}

\begin{proof}
	Stel $L_1$ is niet-beslisbaar en $L_2$ wel. We hebben zonet bewezen dat als $L_2$ beslisbaar is, ook $L_1$ beslisbaar moet zijn. Contradictie.
\end{proof}

\subsubsection*{Orakels en hi\"erarchie van beslisbaarheid ($\leq_T$)}

De tweede notatie heeft betrekking tot orakelmachines in plaats van Turingmachines. Een orakelmachine heeft een andere structuur en werking die deze in staat stelt om, onder andere, $A_{TM}$ te beslissen. Een orakelmachine heeft bezit eigenlijk een grote map van booleans, met elke boolean behorend tot een string. Elke mogelijke string is gekoppeld aan deze $0$ of $1$ waarde. Een orakel dat een taal beslist zet alle strings die tot die taal behoren op $1$, alle andere op $0$. Door een inputstring $s$ te encoderen naar de locatie van de corresponderende booleaanse waarde, kan nagegaan worden of de string tot de taal behoort of niet. Het kan dus nooit in een oneindige lus terecht komen!
\\\\
Het nadeel is echter dat dit enkel een theoretische voorstelling is, die enkel conceptueel gebruikt kan worden. Het is onmogelijk om een bitmap met booleans te implementeren voor elke bestaande string\footnote{Want dat zijn er te veel.}.

\begin{theorem}[Turingreduceerbaar]
	Een taal $A$ is Turingreduceerbaar naar taal $B$, indien $A$ beslisbaar is relatief t.o.v. $B$, t.t.z. er bestaat een orakelachine $O^B$ die $A$ beslist. De notatie is $A \leq_T B$.
\end{theorem}

Dit is inderdaad zeer gelijkend op het eerste deel van deze vraag. In plaats van een beslisser voor $B$ te hebben, die $A$ ook beslist, gebruiken we nu een orakel. Dit orakel is dan (zoals eerder vermeld) een theoretisch hulpmiddel dat we kunnen gebruiken om onze kennis toe te passen op meerdere talen. Deze kunnen we echter in realiteit niet implementeren zoals we net beschreven hebben.

\begin{theorem}
	Indien $A \leq_T B$ en $B$ is beslisbaar, dan is $A$ beslisbaar.
\end{theorem}

\begin{proof}
	De definitie zegt ons dat $A \leq_T B$ enkel geldt indien we een orakel $O^B$ hebben dat $B$ \'en $A$ beslist. Dit is dus volledig afleidbaar van de definitie.\\
	Of anders: stel dat $B$ beslisbaar is en $A$ niet. Dan hebben we een orakel $O^B$ dat (theoretisch) $B$ beslist, maar niet $A$ (want deze is niet beslisbaar). Dit is meteen een contradictie met de definitie.
\end{proof}

\begin{theorem}
	Indien $A \leq_m B$ dan is ook $A \leq_T B$.\\
	M.a.w. $\leq_m$ is fijner dan $\leq_T$.
\end{theorem}

Dit is vanzelfsprekend indien we beseffen dat het orakel (nogmaals) een theoretische uitbreiding is op de Turingmachine. We gebruiken de Turingmachines om talen te herkennen of the beslissen. Het is mogelijk zo een machine te implementeren in een taal naar keuze. Er is echter een grens op het aantal talen dat we kunnen beslissen, aangezien een aantal in een oneindige lus kunnen komen tijdens het beslissingsproces. Dit is in de praktijk een probleem. Een theoretische oplossing daarvoor is het orakel. We kunnen deze machine wel gebruiken om theoretisch verder te redeneren. Dit will zeggen dat het orakel alle talen beslist dit een Turingmachine kan besliseen, plus de talen die een turingmachine niet kan beslissen (oneindig lus). Hierdoor is $A \leq_m B$ fijner dan $A \leq_T B$.

\end{document}